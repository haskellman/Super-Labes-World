% ==============================================================================
% PG - Nome do Aluno
% Capítulo 1 - Introdução
% ==============================================================================
\chapter{Introdução}
\label{sec-intro}
% Talvez, devemos fazer da seguinte forma: falar das dificuldades do lab (no embarque, com o aprendizado, etc...), depois falar sobre as possibilidades de jogos como ferramenta para auxiliar no aprendizado
O Laboratório de Práticas em Engenharia de Software “Ricardo de Almeida Falbo” LabES\footnote{Laboratório de Práticas em Engenharia de Software – LabES \url{https://labes.inf.ufes.br}.}
 é um importante laboratório de extensão vinculado ao  Departamento de Informática do Centro Tecnológico da UFES. Seu objetivo é capacitar estudantes a aplicar métodos, técnicas e procedimentos de ponta em Engenharia de Software, visando aproximar sua formação de necessidades dos diversos setores produtivos além de produzir software a partir de demandas de clientes internos e externos à Universidade \cite{LabES}. Atualmente, o LabES integra cinco projetos ativos, sendo eles:
\begin{itemize}
	\item FAVO: tem como objetivo o desenvolvimento de um aplicativo para fomentar a cultura de inovação nas organizações, onde os usuários podem acompanhar as ações de inovação e participar de programas de ideias de forma \textit{gamificada};
	
	\item Hub Criativo Virtual ou simplesmente Hub ES+: visa a construção de um Hub Criativo Virtual, plataforma online que conectará os espaços físicos do Hub ES+ com o mundo virtual, utilizando recursos de \textit{gamificação} para prover um ambiente imersivo para interação e formação dos criativos e integrando também um painel de dados com as pesquisas do Observatório da Economia Criativa Capixaba e o portal Vitrine Criativa da SECULT-ES;
	
	\item Marvin: tem como objetivo atender solicitações específicas da graduação e pós-graduação do Departamento de Informática da UFES, por exemplo, problemas relacionados a tarefas administrativas vinculadas aos cursos, departamentos e programas da UFES; sendo assim, esse projeto visa a integração das ferramentas existentes e o desenvolvimento de novas ferramentas não cobertas pelos atuais sistemas de informação do NTI/UFES;
	
	\item SEEDES: trata-se do primeiro programa público de aceleração de \textit{startups} no Espírito Santo, um programa do Governo do Estado do Espírito Santo que tem por objetivo fomentar o desenvolvimento de empresas e ideias inovadoras. No LabES, o projeto SEEDES se propõe a desenvolver uma plataforma Web pública para o Ecossistema Capixaba de Inovação com coleta de dados e indicadores sistemáticos do ecossistema local, visando viabilizar um painel com busca automatizada com supervisão na plataforma.;
	
	\item SigAMAES: projeto ao qual este graduando fez parte, que tem como objetivo o desenvolvimento e manutenção de um sistema de informação gerencial que serve de apoio às atividades exercidas pela instituição Associação dos Amigos dos Autistas do Espírito Santo (AMAES)\footnote{AMAES – Associação dos Amigos dos Autistas do Estado do Espírito Santo \url{https://amaes.org.br}.}
, uma instituição sem fins lucrativos, oficialmente constituída em 2001 por pais de autistas e também administrada voluntariamente por eles além de familiares e amigos dos autistas; o sistema tem como principais funcionalidades facilitar o cadastro de informações sobre os autistas, prover mecanismos para gerenciamento dos atendimentos oferecidos pela instituição, oferecer controle de acesso a informações sensíveis e produzir análises estatísticas sobre os dados coletados.;
\end{itemize}
Atualmente, o embarque de novos membros estudantes no laboratório é feito por preenchimento de um formulário no site oficial que vai para um cadastro de reserva que, além das informações pessoais e outras informações sobre disponibilidade, há duas perguntas que dão uma ideia sobre o nível técnico do estudante. Caso o estudante ainda não tenha conversado com algum professor sobre um projeto específico, com as informações do cadastro, cabe aos professores responsáveis pelo laboratório decidir em qual dos atuais cinco projetos do laboratório o estudante será alocado.  Embora essa abordagem seja eficaz para avaliar o interesse do estudante no laboratório, ela apresenta limitações, como: (i) entender o nível de conhecimento técnico do estudante; e (ii) fornecer ao estudante informações importantes sobre o laboratório.


%%% Início de seção. %%%
\section{Motivação e Justificativa}
\label{sec-intro-motjus}
Somando-se às considerações acima, uma caraterística de alguns projetos do LabES é a alta rotatividade da equipe, devido à dificuldade em oferecer bolsas de estudos para todos os estudantes membros. Esse fenômeno gera atrasos no desenvolvimento dos projetos, uma vez que novos membros precisam passar por etapas de treinamentos para que consigam começar a produzir. Quando passam algum tempo no projeto e tornam-se mais experientes e capazes de produzir e ajudar novos membros, os estudantes deixam o projeto, em busca de estágios remunerados. Ou seja, há uma grande dificuldade de manter membros \textit{seniors} nas equipes, que possam ajudar no embarque de novos estudantes. 

Diante desse cenário surge uma necessidade de ajudar esses novos estudantes no seu aprendizado inicial, pavimentando a entrada de novos alunos a fim de acelerar o progresso dos alunos para que sejam capazes de contribuir nos projetos do laboratório.
%há um gap muito grande aqui. Estamos falando de labes, embarque, dificuldades dos projetos do labes... de repente vem algo sobre jogos! Temos que preencher esse gap. Talvez falar algo específico sobre como ajudar no aprendizado com gamificação? Tem que pensar um pouco melhor...
Com os avanços da tecnologia e consequentemente da qualidade de vida, a busca por entretenimento é cada vez mais presente Mediante ao uso frequente de celulares e computadores, as constantes notificações distraem os estudantes, comprometendo seus resultados. Cada vez mais os professores e pedagogos tem se reinventado criado novas estratégias em suas aulas para conseguirem transmitir seu conteúdo de forma eficaz. Por isso, gradativamente, tem-se usado mais da tecnologia para ensinar, e uma das abordagens comumente utilizadas é o uso de jogos. 

Os jogos apresentam-se como um recurso na útil na educação, pois aumentam o interesse dos estudantes nos conteúdos abordados. Um exemplo disso é a \textit{“Quest to Learn}” (Q2L)\footnote{\textit{“Quest to Learn}” (Q2L) \url{https://q2l.org}} uma escola pública de Nova York nos Estados Unidos que, diferentemente das escolas tradicionais, todo seu currículo é baseado nos princípios de design de jogos. Seus resultados de proficiência disciplinar se mostrou superior às demais escolas da mesma região, com destaque em relação à leitura; 27\% a mais \cite{Q2L}.

O Brasil é o décimo maior mercado de games do mundo, com mais de 100 milhões de jogadores.
Em sua ultima pesquisa, a PGB (Pesquisa Game Brasil) apontou que 73,9\% dos brasileiros tem o hábito de jogar video-games \cite{PGB}. Além disso a indústria brasileira de games ainda é muito pequena, existem apenas 1.042 estúdios de desenvolvimento de games ativos em todo país que, se somados, possuem um faturamento estimado de US\$ 252,6 milhões \cite{Abragames}, um pouco menos que 10\% de todo o faturamento da indústria de games mundial no mesmo ano, apontando a necessidade de fomentação do desenvolvimento de jogos locais.

%esse trabalho também pode ser usado como material de apoio a programadores e entusiastas interessados em desenvolvimento de jogos. como o Brasil tem um mercado em potencial nessa industria que movimenta Bilhões,% 

% Outra informação relevante é que segundo a principal plataforma de dados de jogos para PC e console  Newzoo, o Brasil é o décimo maior mercado de games do mundo, com mais de 100 milhões de jogadores que gastaram 2,7 bilhões de dólares em somente em 2022 \cite{NEWZOO}, a PGB (Pesquisa Game Brasil) apontou em sua ultima pesquisa que 73,9\% das pessoas afirmaram ter o hábito de jogar videogames, sendo que 85,4\% destes afirmam ter essa como sua principal fonte de entretenimento \cite{PGB} . Vendo que é uma área atraente aos publico, surgiu a ideia de criar um jogo com propósito educativo,  um game que pudesse  direcionar o aprendizado.


%%% (Associação Brasileira das Desenvolvedoras de Games) divulgado em 2023  %%%

% existem muitos temas a serem explorados, muitos jogos a serem produzidos


% de simulação RPG (Role-playing game) ou simplesmente "jogo de interpretação de papéis",  em português , onde o jogador assume o papel de um estudante que tem o objetivo de entrar no LabES, e assim a história vai levando o estudante a alguns desafios e enquanto direciona o conhecimento forma lúdica sobre aspectos que são consideradas de base para um aluno entrar com o pé direito no projeto.


%%% Início de seção. %%%
\section{Objetivos}
\label{sec-intro-obj}

% Nesta subseção, deve ser descrito o objetivo geral do trabalho, detalhando em seguida, seus objetivos específicos. O \textbf{Objetivo Geral} expressa a finalidade principal do trabalho: para quê? Deve ter coerência direta com o tema do trabalho e ser apresentado em uma frase que inicie com um verbo no infinitivo. O objetivo geral do trabalho está relacionado ao resultado principal do trabalho. Os \textbf{Objetivos Específicos} apresentam os detalhes ou desdobramentos do objetivo geral que levam a resultados intermediários e relevantes para alcançar o objetivo geral. Sempre será mais de um objetivo específico, todos iniciando com verbo no infinitivo.



O Objetivo principal deste projeto de graduação é desenvolver um jogo capaz de apoiar o embarque de novos membros do LabES, esse jogo deve  (i) ser um jogo introdutório que aborda questões básicas e úteis para o começo de seu tempo de colaboração no laboratório, (ii) tenha um feedback ativo e eficaz, ou seja, ao errar, a mecânica proposta fará com que o jogador fique ciente de forma imediata, garantindo um aprendizado acelerado (iii) O jogador seja direcionado a materiais materiais acadêmicos que ensinem de forma organizada e eficiente o que o estudante ainda não tenha aprendido. Estas características tornarão o o jogo não apenas capaz de testar o nível do aluno, mas também elevá-lo.

    São objetivos específicos deste trabalho:
\begin{itemize}
    \item Projetar o sistema;
    \item Realizar o levantamento de requisitos do sistema;
    \item Definir as tecnologias a serem utilizadas;
    \item Desenvolver as questões a serem testadas/ensinadas;
    \item Estruturar um protótipo;
    \item Desenvolver o jogo;
\end{itemize}

%diferentemente de uma prova formal, onde toda essa parte ocorre de forma muito mais lenta, desgastante e trabalhosa para ambas as partes.%


% onde de forma divertida o estudante acaba adquirindo conhecimentos e tendo um direcionamento do seu aprendizado tendo acesso a materiais de estudo que são considerados importantes para começar sua caminhada no LabES e assim começar com o pé direito maximizando a sua curva de aprendizado

% Além disso essa monografia pode ser usada como material de apoio a programadores e entusiastas que gostariam de iniciar no pygame e desenvolvimento de jogos, uma vez que o conteúdo em português sobre game design é ainda escasso. Mais a frente teremos um capítulo dedicado a explicar como e quais ferramentas foram utilizadas no desenvolvimento desse jogo 
% \#\#\# Acrescentar mais aqui depois

% esse trabalho também tem por objetivo ser um material de apoio a programadores e entusiastas interessados em desenvolvimento de jogos.


%%% Início de seção. %%%
\section{Método de Desenvolvimento do Trabalho}
\label{sec-intro-met}
%%%Nesta subseção, deve ser apresentado o \textbf{Método de Desenvolvimento} (ou o \textbf{Método de Pesquisa}, quando for o caso) utilizado no trabalho. Aqui são apresentadas as atividades realizadas e os procedimentos/técnicas que foram usados durante o desenvolvimento do trabalho.%%%
O Desenvolvimento do trabalho começou com a busca de uma resposta fundamental: "Como fazer um jogo educativo ser divertido?". Esse é um dos maiores desafios relacionados aos jogos educativos, torná-los interessantes de forma que o jogador seja incentivado a continuar jogando e, consequentemente, aprendendo. Com essa resposta em mãos, foi realizado um levantamento das principais funcionalidades e mecânicas que o jogo deveria contemplar.

O próximo passo foi a realização de uma pesquisa de \textit{engines}, bibliotecas e ferramentas gerais que seriam utilizadas no desenvolvimento do jogo. Nessa pesquisa foram feitas análises da capacidade de cada ferramenta em reproduzir as funcionalidades definidas na etapa anterior.
Após a definição do diferencial do jogo e as ferramentas a serem utilizadas em sua implementação, iniciou-se a etapa de desenvolvimento.

No início do projeto, o escopo do jogo não estava plenamente definido, por isso o andamento do desenvolvimento foi feito com base mo modelo evolucionário espiral, onde novos requisitos são adquiridos em paralelo à evolução do sistema. Dessa forma, a cada nova versão do jogo eram feitas reuniões com a orientadora e com colegas de curso, que iam sugerindo novas ideias e melhorias.

Na última etapa do projeto, com o jogo já implementado, foi realizada a sua disponibilização aos membros do laboratório, a membros egressos e também a estudantes de computação. Após o teste do jogo, eles eram orientados a preencher um formulário para coletar informações de mecânica e usabilidade do jogo. Estes resultados foram aproveitados na geração das conclusões do projeto, que puderam quantificar seu impacto em um ambiente mais próximo ao real.




%%% Início de seção. %%%
\section{Organização da Monografia}
\label{sec-intro-organizacao}
%%% Além desta introdução, este modelo de monografia é composto por outros cinco capítulos: %%%

Este trabalho está dividido em seis capítulos contando com este, apresentados na seguinte ordem:


\begin{itemize}
	\item O Capítulo~\ref{sec-referencial} apresenta uma fundamentação teórica dos conceitos e tecnologias
 utilizados no trabalho;
	
	\item O Capítulo~\ref{sec-contribuicao} apresenta a principal contribuição do trabalho;

        \item O Capítulo \ref{sec-implementacao} apresenta detalhes da implementação e também as decisões da aplicação;
        
	\item O Capítulo~\ref{sec-avaliacao} apresenta a apresentação e avaliação da proposta;
	
	\item O Capítulo~\ref{sec-conclusoes} apresenta as considerações finais do trabalho;
	
\end{itemize}


