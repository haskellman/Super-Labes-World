% ==============================================================================
% PG - Nome do Aluno
% Capítulo 1 - Introdução
% ==============================================================================
\chapter{Introdução}
\label{sec-intro}
% Talvez, devemos fazer da seguinte forma: falar das dificuldades do lab (no embarque, com o aprendizado, etc...), depois falar sobre as possibilidades de jogos como ferramenta para auxiliar no aprendizado
O Laboratório de Práticas em Engenharia de Software “Ricardo de Almeida Falbo” LabES\footnote{Laboratório de Práticas em Engenharia de Software – LabES \url{https://labes.inf.ufes.br}.}
 é um importante laboratório de extensão vinculado ao Departamento de informática do Centro tecnológico da UFES que tem por objetivo capacitar os(as) estudantes a aplicar métodos, técnicas e procedimentos do estado-da-arte e do estado-da-prática em Engenharia de Software, visando aproximar sua formação de necessidades dos diversos setores produtivos além de produzir software a partir de demandas de clientes internos e externos à Universidade \cite{LabES}. Atualmente, o LabES integra cinco projetos ativos, sendo eles:
\begin{itemize}
	\item FAVO: tem como objetivo o desenvolvimento de um aplicativo para fomentar a cultura de inovação nas organizações, onde os usuários podem acompanhar as ações de inovação e participar de programas de ideias de forma \textit{gamificada};
	
	\item Hub Criativo Virtual ou simplesmente Hub ES+: visa a construção de um Hub Criativo Virtual, plataforma online que conectará os espaços físicos do Hub ES+ com o mundo virtual, utilizando recursos de \textit{gamificação} para prover um ambiente imersivo para interação e formação dos criativos e integrando também um painel de dados com as pesquisas do Observatório da Economia Criativa Capixaba e o portal Vitrine Criativa da SECULT-ES;
	
	\item Marvin: tem como objetivo atender solicitações específicas da graduação e pós-graduação do Departamento de Informática da UFES, por exemplo, problemas relacionados a tarefas administrativas vinculadas aos cursos, departamentos e programas da UFES; sendo assim, esse projeto visa a integração das ferramentas existentes e o desenvolvimento de novas ferramentas não cobertas pelos atuais sistemas de informação do NTI/UFES;
	
	\item SEEDES: trata-se do primeiro programa público de aceleração de \textit{startups} no Espírito Santo, um programa do Governo do Estado do Espírito Santo que tem por objetivo fomentar o desenvolvimento de empresas e ideias inovadoras. No LabES, o projeto SEEDES se propõe a desenvolver uma plataforma Web pública para o Ecossistema Capixaba de Inovação com coleta de dados e indicadores sistemáticos do ecossistema local, visando viabilizar um painel com busca automatizada com supervisão na plataforma.;
	
	\item SigAMAES: projeto ao qual este graduando fez parte, que tem como objetivo o desenvolvimento e manutenção de um sistema de informação gerencial que serve de apoio às atividades exercidas pela instituição Associação dos Amigos dos Autistas do Espírito Santo (AMAES)\footnote{AMAES – Associação dos Amigos dos Autistas do Estado do Espírito Santo \url{https://amaes.org.br}.}
, uma instituição sem fins lucrativos, oficialmente constituída em 2001 por pais de autistas e também administrada voluntariamente por eles além de familiares e amigos dos autistas; o sistema tem como principais funcionalidades facilitar o cadastro de informações sobre os autistas, prover mecanismos para gerenciamento dos atendimentos oferecidos pela instituição, oferecer controle de acesso a informações sensíveis e produzir análises estatísticas sobre os dados coletados.;
\end{itemize}
Atualmente, o embarque de novos membros estudantes no laboratório é feito por preenchimento de um formulário no site oficial que vai para um cadastro de reserva que, além das informações pessoais e outras informações sobre disponibilidade, há duas perguntas que dão uma ideia sobre o nível técnico do estudante. Caso o estudante ainda não tenha conversado com algum professor sobre um projeto específico, com as informações do cadastro, cabe aos professores responsáveis pelo laboratório decidir em qual dos atuais cinco projetos do laboratório o estudante será alocado. Essa abordagem é eficaz para indicar o interesse do estudante no laboratório; contudo, ela ainda não é a ideal para: (i) entender o nível de conhecimento técnico do estudante; e (ii) fornecer ao estudante informações importantes sobre o laboratório.


%%% Início de seção. %%%
\section{Motivação e Justificativa}
\label{sec-intro-motjus}
Somando-se às considerações acima, uma caraterística de alguns projetos do LabES é a alta rotatividade da equipe, devido à dificuldade em oferecer bolsas de estudos para todos os estudantes membros. Esse fenômeno gera atrasos no desenvolvimento dos projetos uma vez que novos membros precisam passar por etapas de treinamentos para que consigam começar a produzir. Quando passam algum tempo no projeto e tornam-se mais experientes e capazes de produzir e ajudar novos membros, os estudantes deixam o projeto, em busca de estágios remunerados. Ou seja, há uma grande dificuldade de manter membros \textit{seniors} nas equipes, que possam ajudar no embarque de novos estudantes. 

Diante desse cenário surge uma necessidade de ajudar esses novos estudantes no seu aprendizado inicial, pavimentando a entrada de novos alunos a fim de acelerar o seu progresso para que sejam capazes de contribuir nos projetos do laboratório.
%há um gap muito grande aqui. Estamos falando de labes, embarque, dificuldades dos projetos do labes... de repente vem algo sobre jogos! Temos que preencher esse gap. Talvez falar algo específico sobre como ajudar no aprendizado com gamificação? Tem que pensar um pouco melhor...
Com os avanços da tecnologia e consequentemente da qualidade de vida, a busca por entretenimento é cada vez mais presente, o uso cada vez mais frequente de celulares e computadores onde existem milhares de notificações que distraem os estudantes comprometem seus resultados é real, cada vez mais os professores e pedagogos tem se reinventado e criam novas estratégias em suas aulas para conseguirem transmitir seu conteúdo de forma eficaz, por isso gradativamente tem-se usado mais da tecnologia para ensinar, uma abordagem é através de jogos. 

Os jogos apresentam-se como um recurso na qual professores podem tornar suas aulas mais atraentes e prazerosas para os estudantes, um exemplo disso é a “Quest to Learn” (Q2L)\footnote{“Quest to Learn” (Q2L) \url{https://q2l.org}} uma escola pública de Nova York nos Estados Unidos, que diferente das escolas tradicionais todo seu currículo é baseado nos princípios de design de jogos, e seus resultados em proeficiência nas disciplinas foi superior as demais escolas na mesma região com destaque a leitura, que foi 27\% a mais \cite{Q2L}.

O Brasil é o décimo maior mercado de games do mundo, com mais de 100 milhões de jogadores.
Em sua ultima pesquisa, a PGB (Pesquisa Game Brasil) apontou que 73,9\% dos brasileiros tem o hábito de jogar video-games \cite{PGB}. Além disso a indústria brasileira de games ainda é muito pequena, existem apenas 1.042 estúdios de desenvolvimento de games ativos em todo país, que se somados tem um faturamento estimado de US\$ 252,6 milhões \cite{Abragames}, um pouco menos que 10\% de todo o faturamento da industria de games no mesmo ano, sendo assim também é importante fomentar o desenvolvimento de jogos locais.

%esse trabalho também pode ser usado como material de apoio a programadores e entusiastas interessados em desenvolvimento de jogos. como o Brasil tem um mercado em potencial nessa industria que movimenta Bilhões,% 

% Outra informação relevante é que segundo a principal plataforma de dados de jogos para PC e console  Newzoo, o Brasil é o décimo maior mercado de games do mundo, com mais de 100 milhões de jogadores que gastaram 2,7 bilhões de dólares em somente em 2022 \cite{NEWZOO}, a PGB (Pesquisa Game Brasil) apontou em sua ultima pesquisa que 73,9\% das pessoas afirmaram ter o hábito de jogar videogames, sendo que 85,4\% destes afirmam ter essa como sua principal fonte de entretenimento \cite{PGB} . Vendo que é uma área atraente aos publico, surgiu a ideia de criar um jogo com propósito educativo,  um game que pudesse  direcionar o aprendizado.


%%% (Associação Brasileira das Desenvolvedoras de Games) divulgado em 2023  %%%

% existem muitos temas a serem explorados, muitos jogos a serem produzidos


% de simulação RPG (Role-playing game) ou simplesmente "jogo de interpretação de papéis",  em português , onde o jogador assume o papel de um estudante que tem o objetivo de entrar no LabES, e assim a história vai levando o estudante a alguns desafios e enquanto direciona o conhecimento forma lúdica sobre aspectos que são consideradas de base para um aluno entrar com o pé direito no projeto.


%%% Início de seção. %%%
\section{Objetivos}
\label{sec-intro-obj}

% Nesta subseção, deve ser descrito o objetivo geral do trabalho, detalhando em seguida, seus objetivos específicos. O \textbf{Objetivo Geral} expressa a finalidade principal do trabalho: para quê? Deve ter coerência direta com o tema do trabalho e ser apresentado em uma frase que inicie com um verbo no infinitivo. O objetivo geral do trabalho está relacionado ao resultado principal do trabalho. Os \textbf{Objetivos Específicos} apresentam os detalhes ou desdobramentos do objetivo geral que levam a resultados intermediários e relevantes para alcançar o objetivo geral. Sempre será mais de um objetivo específico, todos iniciando com verbo no infinitivo.



O Objetivo principal deste projeto de graduação é desenvolver um jogo capaz de apoiar o embarque de novos membros do LabES, esse jogo deve  (i) ser um jogo introdutório que aborda questões de base que sejam uteis para o inicio do seu tempo no laboratório (ii) tenha um rápido feedback de resposta, ou seja, uma vez que o jogador erra ele rapidamente fica ciente disso a fim de acelerar o aprendizado (iii) o jogador seja direcionado a materiais com o propósito de ensinar oque o estudante ainda não saiba assim sendo um jogo não apenas capaz de testar o nível do aluno, mas também capaz de elevá-lo.

    São objetivos específicos deste trabalho:
\begin{itemize}
    \item Projetar o sistema;
    \item Realizar o levantamento de requisitos do sistema;
    \item Definir as tecnologias a serem utilizadas;
    \item Desenvolver as questões a serem testadas/ensinadas;
    \item Estruturar um protótipo;
    \item Desenvolver o jogo;
\end{itemize}

%diferentemente de uma prova formal, onde toda essa parte ocorre de forma muito mais lenta, desgastante e trabalhosa para ambas as partes.%


% onde de forma divertida o estudante acaba adquirindo conhecimentos e tendo um direcionamento do seu aprendizado tendo acesso a materiais de estudo que são considerados importantes para começar sua caminhada no LabES e assim começar com o pé direito maximizando a sua curva de aprendizado

% Além disso essa monografia pode ser usada como material de apoio a programadores e entusiastas que gostariam de iniciar no pygame e desenvolvimento de jogos, uma vez que o conteúdo em português sobre game design é ainda escasso. Mais a frente teremos um capítulo dedicado a explicar como e quais ferramentas foram utilizadas no desenvolvimento desse jogo 
% \#\#\# Acrescentar mais aqui depois

% esse trabalho também tem por objetivo ser um material de apoio a programadores e entusiastas interessados em desenvolvimento de jogos.


%%% Início de seção. %%%
\section{Método de Desenvolvimento do Trabalho}
\label{sec-intro-met}
%%%Nesta subseção, deve ser apresentado o \textbf{Método de Desenvolvimento} (ou o \textbf{Método de Pesquisa}, quando for o caso) utilizado no trabalho. Aqui são apresentadas as atividades realizadas e os procedimentos/técnicas que foram usados durante o desenvolvimento do trabalho.%%%
O Desenvolvimento do trabalho começou com a busca para a uma resposta fundamental "Como fazer um jogo educativo ser divertido?", esse é um dos maiores desafios relacionados a jogos educativos, torna-los interessantes a fim de que o jogador queira continuar jogando e consequentemente aprendendo, com essa resposta em mãos foi feito um levantamento das principais funcionalidades e mecânicas que o jogo deve contemplar.

    Após isso foi feita uma pesquisa de engines, bibliotecas e as ferramentas num todo que seriam utilizadas no desenvolvimento do jogo e que sejam capazes de reproduzir as funcionalidades da etapa anterior,
    com isso definido começou o desenvolvimento do jogo.

Como nem todo o escopo do jogo estava completamente definido no início do projeto de graduação, o andamento do projeto foi feito com base mo modelo evolutivo, onde novos requisitos são adquiridos em paralelo à evolução do sistema; assim, a cada nova versão do game foram feitas reuniões com a orientadora e também com colegas de curso nas quais iam surgindo novas ideias e melhorias.

Por fim com o jogo já concluído, ele será distribuído aos alunos já membros do laboratório e também a outros estudantes de computação seguido de um formulário a fim de coletar informações sobre e gerar conclusões. Essa etápa é fundamental a fim de conseguir quantificar o impacto que o trabalho gerou.




%%% Início de seção. %%%
\section{Organização da Monografia}
\label{sec-intro-organizacao}
%%% Além desta introdução, este modelo de monografia é composto por outros cinco capítulos: %%%

Este trabalho está dividido em cinco capítulos com este, apresentados na seguinte ordem:


\begin{itemize}
	\item O Capítulo~\ref{sec-referencial} apresenta uma fundamentação teórica dos conceitos e tecnologias
 utilizados no trabalho;
	
	\item O Capítulo~\ref{sec-contribuicao} apresenta a principal contribuição do trabalho;
	
	\item O Capítulo~\ref{sec-avaliacao} apresenta a avaliação da proposta, quando a mesma tiver sido realizada e requeira uma descrição detalhada;
	
	\item O Capítulo~\ref{sec-conclusoes} apresenta as considerações finais do trabalho;
	
	\item O Capítulo~\ref{sec-dicaslatex} traz dicas básicas para escrita de textos científicos em \latex.
\end{itemize}


