% ==============================================================================
% PG - Nome do Aluno
% Capítulo 5 - Considerações Finais
% ==============================================================================
\chapter{Conclusão}
\label{sec-conclusoes}


% Neste capítulo devem ser realizadas as considerações finais do trabalho, sendo apresentadas suas principais contribuições, limitações, lições aprendidas durante o desenvolvimento do trabalho, dificuldades enfrentadas e perspectivas de trabalhos futuros. O capítulo deve ter entre 3 e 5 páginas.


%%% Início de seção. %%%
\section{Considerações Finais}
\label{sec-conclusoes-consideracoes}

% Esta seção deve apresentar um texto de fechamento do trabalho, devendo incluir considerações sobre o trabalho desenvolvido, suas limitações, contribuições, experiência adquirida pelo aluno e lições aprendidas ao longo do desenvolvimento, bem como dificuldades enfrentadas durante o desenvolvimento do trabalho. Nesta seção é preciso mostrar claramente a relação entre os resultados produzidos no trabalho e os objetivos estabelecidos no Capítulo~\ref{sec-intro} 

Neste projeto de graduação, foi desenvolvido o jogo \textit{Super Labes World}, cujo objetivo é auxiliar os novos integrantes do LabES a compreenderem, ou revisarem, assuntos pertinentes ao LabES, a fim de melhorarem a integração nos projetos. O jogo apresenta essa proposta de maneira interativa e lúdica, proporcionando uma experiência didática descontraída. No jogo, o jogador interpreta um aluno da UFES que tem como objetivo entrar no SigAMAES, um dos projetos do LabES. O objetivo no jogo é vencer os desafios propostos pelos professores e adquirir as três \textit{chaves}; essas chaves comprovam de que o aluno possui aquele conhecimento. Ao conseguir todas as chaves, o jogador pode então ter acesso ao LabES.

% Apesar do foco do jogo ser os membros do LabES nessa versão, o jogo pode facilmente ser continuado e adaptado para diferentes contextos e áreas da UFES. O código do jogo está disponível publicamente junto com essa monografia que pode ser utilizada como referencial para programadores e \textit{game designers}. 

Este trabalho de conclusão de curso também pode ser relevante para interessados e entusiastas na área de \textit{game design}, uma vez que essa monografia também discutiu tecnologias, ferramentas e a metodologia usada no desenvolvimento de Super LabES World. Esses conceitos podem ser aplicados no desenvolvimento de jogos no geral. O código-fonte do jogo está disponível publicamente, junto com essa monografia, no Github \footnote{Super Labes World no Github \url{https://github.com/haskellman/Super-Labes-World/blob/main/README.md}}.

O uso das bibliotecas Pygame e Pytmx permitiram usar várias camadas de abstração, proporcionando um desenvolvimento eficiente e ágil do jogo. Os softwares Tiled e Aseprite também cumpriram um papel importante para a criação do jogo; o uso de suas documentações foram essenciais para rápido entendimento das ferramentas. Foi possível aplicar boas práticas de programação e de engenharia de software, aprendidas durante a graduação, que resultaram num código altamente flexível e aberto a alterações e expansões.

Reavaliando os objetivos dessa monografia, descritos na seção \ref{sec-intro}, consideramos que foram integralmente alcançados. Os resultados alcançados com a finalização do jogo foram extremamente positivas nesse contexto, o uso de jogos para educação se mostrou uma ferramenta com alto potencial de ensino, isso pode ser comprovado na seção \ref{sec:avaliacao-do-jogo}, onde foi detalhado os resultados obtidos com o jogo. 

Pensando nas dificuldades enfrentadas durante o processo de criação do jogo, a primeira a ser superada foi a curva de aprendizado. É comum ao desenvolver algo novo enfrentar dificuldades, ter que aprender novas ferramentas, bibliotecas, \textit{frameworks} mas, com uma base sólida, que foi adquirida durante a graduação, foi possível superar os desafios. A segunda foi o tempo: o desenvolvimento de um jogo exige diversas competências interdisciplinares. No mercado, profissionais especializados são contratados para cuidar de cada aspecto do jogo, como falado na seção \ref{sec-desenvolvimento-de-jogos}. Mesclar todas essas habilidades neste trabalho foi um grande desafio.

Com uma ideia inicialmente enorme e as dificuldades citadas no parágrafo anterior para com o desenvolvimento do jogo, os conhecimentos obtidos ao longo do curso foram fundamentais. A busca por ``dividir e conquistar'' os problemas que surgiram no caminho foi a estratégia utilizada com sucesso.


%%% Início de seção. %%%
\section{Trabalhos Futuros}
\label{sec-conclusoes-trabalhosfuturos}

% Nesta seção devem ser identificados trabalhos futuros que poderão ser realizados a partir dos resultados obtidos até o momento no trabalho. Idealmente, trabalhos futuros não devem apenas ser citados. Recomenda-se discutir aspectos sobre como podem ser realizados e por que é importante que sejam realizados (que benefícios podem ser obtidos com sua realização).

O \textit{Super Labes World} apresenta grande potencial para crescimento e expansão, visto que é um jogo do gênero RPG e sua ambientação não foi totalmente explorada. O jogo possui três professores e, consequentemente, três batalhas. Porém, é possível adicionar novos personagens e novas batalhas. Algumas ideias que surgiram durante o processo de desenvolvimento do jogo e que não foram possíveis de serem implementadas em tempo são:

\begin{itemize}
    \item Um sistema de moedas, onde o jogador ganha recompensas ao vencer batalhas que poderão serem trocadas por recompensas em outros mapas, por exemplo na cantina;
    \item Novos mapas, muitos locais da UFES poderiam ser adicionados;
    \item Novos personagens, o acréscimo de novos personagens seria necessária;
    \item Um sistema de customização de personagem ao iniciar o jogo;
    \item Adição de um mini-mapa, indicando o jogador áreas que não foram exploradas;
    \item Desenvolvimento de um sistema de \textit{save game}, para que o jogador não precise terminar o jogo de uma vez;
    \item Novos itens;
    \item Novos materiais de estudo;
    \item O sistema de adição de perguntas aos personagens poderia ser através de planilhas, para facilitar a alteração de questões para todos os tipos de usuários;
    \item Novas mecânicas de jogo;
    \item Tornar o Super Labes World em um MMO \textit{(Massively Multiplayer Online)} ou seja multijogadores massivos online. Esse que é um gênero de video-game no qual é possível ser jogado online com outros jogadores simultaneamente, podendo ter interação uns com os outros.
\end{itemize}

Quanto à implementação, alguns aspectos poderiam ser aprimorados, como desenvolvimento de um banco de dados em vez  do uso de dicionários para armazenar as informações do jogo. Isso estruturaria melhor o código à medida que o jogo cresce.

A adição de músicas-tema e efeitos sonoros autorais seria muito bem-vinda. Como observado durante a etapa de avaliação de resultados, as músicas e os efeitos sonoros foram considerados muito importantes para despertar o interesse no jogo por seis das sete pessoas avaliadas. Na versão atual, o jogo utilizou músicas já existentes de outros jogos.
