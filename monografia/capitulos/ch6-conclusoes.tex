% ==============================================================================
% PG - Nome do Aluno
% Capítulo 5 - Considerações Finais
% ==============================================================================
\chapter{Conclusão}
\label{sec-conclusoes}


% Neste capítulo devem ser realizadas as considerações finais do trabalho, sendo apresentadas suas principais contribuições, limitações, lições aprendidas durante o desenvolvimento do trabalho, dificuldades enfrentadas e perspectivas de trabalhos futuros. O capítulo deve ter entre 3 e 5 páginas.


%%% Início de seção. %%%
\section{Considerações Finais}
\label{sec-conclusoes-consideracoes}

% Esta seção deve apresentar um texto de fechamento do trabalho, devendo incluir considerações sobre o trabalho desenvolvido, suas limitações, contribuições, experiência adquirida pelo aluno e lições aprendidas ao longo do desenvolvimento, bem como dificuldades enfrentadas durante o desenvolvimento do trabalho. Nesta seção é preciso mostrar claramente a relação entre os resultados produzidos no trabalho e os objetivos estabelecidos no Capítulo~\ref{sec-intro} 

Neste projeto de graduação, foi desenvolvido o jogo \textit{Super Labes World}, cujo objetivo é auxiliar os novos integrantes do LabES a compreenderem, ou revisarem caso já saibam assuntos pertinentes ao LabES, afim de melhorar sua integração nos projetos. O jogo apresenta essa proposta de maneira interativa e lúdica, proporcionando uma experiência didática descontraída. No jogo o jogador interpreta um aluno da UFES que tem como objetivo entrar no SigAMAES, um dos projetos do LabES. O objetivo do jogo é vencer os desafios propostos pelos professores e adquirir as três chaves, essas chaves são uma prova de que o aluno possui aquele conhecimento. Ao jogador conseguir todas essas chaves o jogador pode então ter acesso ao LabES.



% Apesar do foco do jogo ser os membros do LabES nessa versão, o jogo pode facilmente ser continuado e adaptado para diferentes contextos e áreas da UFES. O código do jogo está disponível publicamente junto com essa monografia que pode ser utilizada como referencial para programadores e \textit{game designers}. 

Este trabalho de conclusão de curso também pode ser relevante para interessados e entusiastas na área de \textit{game design}, uma vez que essa monografia apresentou e detalhou tecnologias, ferramentas e a metodologia usada no desenvolvimento de jogo. Esses conceitos podem ser aplicados a diferentes jogos. O código fonte do jogo está disponível publicamente junto com essa monografia no Github.

O uso das bibliotecas Pygame e Pytmx permitiram abstrair várias camadas de abstração proporcionando um desenvolvimento eficiente e ágil do jogo. Os softwares Tiled e Aseprite também cumpriram um papel importante para a criação desse \textit{game}, o uso de suas documentações foram essenciais para rápido entendimento das ferramentas. Foi possível aplicar boas práticas de programação e de engenharia de software, aprendidas durante a graduação, que resultaram num código altamente flexível e aberto à alterações e expansões.

Reavaliando os objetivos dessa monografia, descritos na seção \ref{sec-intro}, consideramos que foram integralmente alcançados. Os resultados alcançados com a finalização do jogo foram extremamente positivos nesse contexto, o uso de jogos para educação se mostrou uma ferramenta com alto potencial de ensino e . Isso pode ser comprovado na seção \ref{}

Pensando nas dificuldades enfrentadas durante o processo de criação do jogo, a primeira a ser superada foi a curva de aprendizado . É comum ao desenvolver algo novo enfrentar dificuldades, ter que aprender novas ferramentas, bibliotecas, \textit{frameworks}, mas, com base sólida que o foi adquirida durante o curso foi possível vencer. A segunda foi o tempo, o desenvolvimento de um jogo exige diversas competências interdisciplinares. No mercado, profissionais especializados são contratados para cuidar de cada aspecto do jogo, como falado na seção \ref{sec-desenvolvimento-de-jogos}, mesclar todas essas habilidades neste trabalho foi um desafio.

Com uma ideia inicialmente enorme e as dificuldades citadas no parágrafo anterior para com o desenvolvimento do jogo, os conhecimentos obtidos ao longo do curso foram fundamentais. A busca por ''dividir e conquistar'' os problemas que surgiram no caminho foi a estratégia utilizada e com resultados.

Finalmente, entendemos que esse projeto final de graduação foi bem-sucedido.

%%% Início de seção. %%%
\section{Trabalhos Futuros}
\label{sec-conclusoes-trabalhosfuturos}

% Nesta seção devem ser identificados trabalhos futuros que poderão ser realizados a partir dos resultados obtidos até o momento no trabalho. Idealmente, trabalhos futuros não devem apenas ser citados. Recomenda-se discutir aspectos sobre como podem ser realizados e por que é importante que sejam realizados (que benefícios podem ser obtidos com sua realização).

O \textit{Super Labes World} apresenta grande potencial para crescimento e expansão, visto que é um jogo do gênero RPG e sua ambientação não foi totalmente explorada. O jogo possui três professores e consequentemente três batalhas, porém, é possível adicionar novos personagens e consequentemente novas batalhas. Algumas ideias que surgiram durante o processo de desenvolvimento do jogo foram.

\begin{itemize}
    \item Um sistema de moedas, onde o jogador ganha recompensas ao vencer batalhas que poderão serem trocadas por recompensas em outros mapas, por exemplo na cantina;
    \item Novos mapas, muitos locais da UFES poderiam ser adicionados;
    \item Novos personagens, o acréscimo de novos personagens seria necessária;
    \item Um sistema de customização de personagem ao iniciar o jogo;
    \item Adição de um mini-mapa, indicando o jogador áreas que não foram exploradas;
    \item Desenvolvimento de um sistema de \textit{save game}, para que o jogador não precise terminar o jogo de uma vez;
    \item Novos items;
    \item Novos materiais de estudo;
    \item O sistema de adição de perguntas aos personagens poderia ser através de planilhas, para facilitar a alteração de questões para todos os tipos de usuários;
    \item Novas mecânicas de jogo;
    \item Tornar o Super Labes World em um MMO \textit{(Massively Multiplayer Online)} ou seja multijogadores massivos online;

\end{itemize}